\documentclass[12pt, xcolor=table]{beamer}
\usepackage{graphicx}
\usepackage[ngerman]{babel}
\usepackage[utf8]{inputenc}
%\usepackage[T1]{fontenc}
\usepackage{amsmath}
\usepackage{amssymb}
\usepackage{listings}
\usepackage{hyperref}
\usepackage{fancyvrb}
\usepackage{color}
\usepackage{verbatim}
\usepackage{alltt}

\usepackage[percent]{overpic}
\usepackage[footnotesize, bf]{caption}
\usepackage[all]{xy}
\input{theme.tex}
\input{syntax}
\renewcommand{\footnotesize}{\tiny}

\begin{document}
\title{Mapping Between Categories: On Functors in Functional Programming}
\author{Lusy}
\date{\today}

\begin{frame}
    \titlepage
    \begin{block}
        \tiny \url{https://github.com/lusy/functorsTalk}
    \end{block}
\end{frame}

\section{Introduction} % (fold)
\begin{frame}
     \frametitle{Introduction}
     % Find a picture to backup the introductory questions
     %* how many people have already seen/done some functional programming?
     %* how many of them are familiar with Haskell?
     %* does anybody know what Functors are?
     \begin{center}
         \includegraphics[scale=0.4]{figures/pickme.png}
     \end{center}
\end{frame}

%\begin{comment}
%\begin{frame}
%    \frametitle{Why is functional programming awesome?}
%    \begin{itemize}
%        \item immutability %excel picture
%        \item concurrency %dinning philosophers
%        \item elegancy
%        %\item first class functions
%    \end{itemize}
%    % code example
%    % excel example for the immutability
%\end{frame}
%\ende{comment}

%\begin{frame}
%    \frametitle{Why is functional programming awesome?}
%    \begin{block}{Immutability + Concurrency}
%        \begin{center}
%            \includegraphics[scale=0.4]{figures/Spreadsheet.png}
%        \end{center}
%    \end{block}
    % code example
    % excel example for the immutability
%\end{frame}

%\begin{frame}
%    \frametitle{Why is functional programming awesome?}
%    \begin{block}{Concurrency}
%        \begin{center}
%            \includegraphics[scale=0.3]{figures/DiningPhilosophers.png}
%        \end{center}
%    \end{block}
    % code example
    % excel example for the immutability
%\end{frame}

%\begin{frame}
%    \frametitle{Why is functional programming awesome?}
%    \begin{block}{Elegancy}
%        \begin{block}{Haskell}
%                \begin{alltt}
%                    \begin{tiny}
%                    qsort :: Ord a => [a] -> [a]
%                    \newline
%                    qsort []     = []
%                    \newline
%                    qsort (x:xs) = qsort (filter (< x) xs) ++ [x] ++ qsort (filter (>= x) xs)
%                \end{tiny}
%                \end{alltt}
%        \end{block}
%        \begin{block}{Javascript}
%                \begin{alltt}
%                    \begin{tiny}
%                    function qsort(a) \{ \\
%                        if (a.length == 0) return []; \\
%
%                        var left = [], right = [], pivot = a[0];\\
%
%                        for (var i = 1; i < a.length; i++) \{ \\
%                            a[i] < pivot ? left.push(a[i]) : right.push(a[i]); \\
%                        \} \\
%
%                        return qsort(left).concat(pivot, qsort(right)); \\
%                    \}
%                    \end{tiny}
%                \end{alltt}
%        \end{block}
%    \end{block}
%\end{frame}

\begin{frame}
    \frametitle{Some Maths}
        \begin{block}{Category Theory}
        \begin{itemize}
            \item mathematical abstraction
            \item categories are sets, vector spaces, or types for computer science..
        \end{itemize}
        \end{block}
\end{frame}

\begin{frame}
    \frametitle{Category Theory 2}
    \begin{itemize}
        \item there are mappings between them: an object can be ``tranferred'' from one category to another
        \item  mappings are structures preservent%: we can ``map'' our data to another category and do some operations on it (because it's easier in this one)
    \end{itemize}
\end{frame}

\section{Functors Definition}
\begin{frame}
    \frametitle{So what are functors?}
    Structure preservent mappings between categories!
    \newline
    \newline
    ``A functor associates to every object of one category an object of another category, and to every morphism in the first category a morphism in the second.''
\end{frame}

\begin{frame}
    \frametitle{The Haskell definition}
    ``Types that can act like a box can be functors. You can think of a list as a box that can be empty or have something inside it, including another box!''
    \vspace{30px}
    \begin{columns}
        \column{0.5\textwidth}
            \begin{center}
            \includegraphics[scale=0.4]{figures/list.png}
            \end{center}
        \column{0.5\textwidth}
            \begin{center}
            \includegraphics[scale=0.4]{figures/maybe.png}
            \end{center}
    \end{columns}
\end{frame}

\section{Haskell Functors}
\begin{frame}
    \frametitle{The Haskell Definition 2}
    Functor is a type class. Very much like Eq, Ord, Show, ...
    It requires a type constructor that takes \textbf{one} parameter

    \begin{alltt}
        > class  Functor    f   where \\
        >       fmap         ::   (a $\to$ b) $\to$ f a $\to$ f b
%               \newline
%               \newline
%        fmap id           =  id \\
%        fmap (g . f)      =  fmap g . fmap f
    \end{alltt}
    \begin{center}
        \includegraphics[scale=0.4]{figures/fmapdef.png}
    \end{center}
\end{frame}

\begin{frame}
    \frametitle{An Example - Lists}

    \begin{alltt}
        > map :: (a $\to$ b) $\to$ [a] $\to$ [b] \\
        > instance Functor [] where \\
        >   fmap = map
    \end{alltt}

    \begin{center}
        \includegraphics[scale=0.3]{figures/fmapList.png}
    \end{center}
\end{frame}

\begin{frame}{Demo!}
        \begin{alltt}
            > fmap (2 +) [1,2,3]
        \end{alltt}
\end{frame}

\begin{comment}
Further examples

Maybe
-----
> data Maybe x = Just x | Nothing

> instance Functor Maybe where
>   fmap f (Just x) = Just (f x)
>   fmap f Nothing = Nothing


> fmap (++ "ho") (Just "hey")

Trees
-----
> data Tree x = Empty | Node x (Tree x) (Tree x)

> instance Functor Tree where
>   fmap f Empty = Empty
>   fmap f (Node x left right) = Node (f x) (fmap f left) (fmap f right)
\end{comment}

\begin{frame}
    \frametitle{You know this already!!}
    % ruby / javascript examples
    \begin{block}{Ruby}
        \begin{alltt}
            [1, 2, 3].map \{ |n| n * n \}
            \newline
            \newline
            => [1, 4, 9]
        \end{alltt}
    \end{block}
\end{frame}

\begin{frame}
    \frametitle{You know this already!!}
    % ruby / javascript examples
    \begin{block}{Javascript}
        \begin{alltt}
            \_.map([1, 2, 3], function(num)\{return num*3;\});
            \newline
            \newline
            => [3, 6, 9]
        \end{alltt}
    \end{block}
\end{frame}

\begin{frame}
    \frametitle{From a functor to another functor}
    \begin{block}{Tree vs Array}
    \begin{columns}
        \column{0.5\textwidth}
            \begin{center}
            \includegraphics[scale=0.3]{figures/tree.png}
            \end{center}
        \column{0.5\textwidth}
            \begin{center}
                [1, 3, 4, 6, 7, 8, 10, 13, 14]
            \end{center}
    \end{columns}
\end{block}
\end{frame}

\begin{frame}
    \frametitle{The Natural Transformation}
    % Mapping between functors
            \begin{equation*}
                   \xymatrix@C+2em@R+2em{
                      F(A) \ar[r]^{fmap\_f} \ar[d]_{\alpha_A} & F(B) \ar[d]^{\alpha_B} \\
                      G(A) \ar[r]_{fmap\_g} & G(B)
                     }
            \end{equation*}

%            $\alpha :: F a \ar G a$
%            Parametricity:
%            $\alpha . fmap\_f = fmap\_f . \alpha$

\end{frame}


\section{The end..} % (fold)
\begin{frame}
    \frametitle{The End...}
    \begin{block}{Thanks!}
        Questions?
    \end{block}
\end{frame}

\section{Picture Sources}
\begin{frame}
    \frametitle{Some Sources}
    \begin{itemize}
        \item \tiny{http://learnyouahaskell.com/}
        \item \tiny{http://adit.io/posts/2013-04-17-functors,\_applicatives,\_and\_monads\_in\_pictures.html}
        \item \tiny{https://en.wikipedia.org/wiki/File:Dining\_philosophers.png}
        \item \tiny{https://commons.wikimedia.org/wiki/File:Spreadsheet.png}
        \item \tiny{https://en.wikibooks.org/wiki/Algorithm\_Implementation/Sorting/Quicksort}
        \item \tiny{https://en.wikipedia.org/wiki/File:Binary\_search\_tree.svg}
        \item \tiny{https://en.wikipedia.org/wiki/Category\_theory}
    \end{itemize}
\end{frame}

\end{document}
