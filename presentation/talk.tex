\documentclass[12pt, xcolor=table]{beamer}
\usepackage{graphicx}
\usepackage[ngerman]{babel}
\usepackage[utf8]{inputenc}
\usepackage{amsmath}
\usepackage{amssymb}
\usepackage{listings}
\usepackage{hyperref}
\usepackage{fancyvrb}
\usepackage{color}
\usepackage{verbatim}
\usepackage{alltt}

\usepackage[percent]{overpic}
\usepackage[footnotesize, bf]{caption}
\input{theme.tex}
\input{syntax}
\renewcommand{\footnotesize}{\tiny}

\begin{document}
\title{Mapping Between Categories: On Functors in Functional Programming}
\author{Lusy}
\date{\today}

\begin{frame}
    \titlepage
    \begin{block}
        \tiny \url{https://github.com/lusy/functorsTalk}
    \end{block}
\end{frame}

\section{Introduction} % (fold)
\begin{frame}
     \frametitle{Introduction}
     % Find a picture to backup the introductory questions
     %* how many people have already seen/done some functional programming?
     %* how many of them are familiar with Haskell?
     %* does anybody know what Functors are?
\end{frame}

\begin{frame}
    \frametitle{Why is functional programming awesome?}
    \begin{itemize}
        \item immutability
        \item concurrency
        \item elegancy
        %\item first class functions
    \end{itemize}
    % code example
    % excel example for the immutability
\end{frame}


\begin{frame}
    \frametitle{Some Maths...}
    What are categories in maths?
    \begin{itemize}
        \item sets
        \item vector spaces
        \item types in programming languages
    \end{itemize}
    %TODO pictures
\end{frame}

\begin{frame}
    \frametitle{Some Maths 2...}
    And we can map between these categories!
    % pictures!
    The mappings are structure preservent...
\end{frame}

\section{Functors Definition}
\begin{frame}
    \frametitle{So what are functors?}
    Structure preservent mappings between categories!
    % example!
\end{frame}

\begin{frame}
    \frametitle{And what are they good for?}
    Motivation

    %\begin{center}
    %    \includegraphics[scale=0.35]{figures/IntroToLDA.png}
    %\end{center}
\end{frame}

\begin{frame}
    \frametitle{The Haskell definition}
    ``Types that can act like a box can be functors. You can think of a list as a box that can be empty or have something inside it, including another box!''
    %pictures!!
\end{frame}

\section{Haskell Functors}
\begin{frame}
    \frametitle{The Haskell Definition 2}
    Functor is a type class. Very much like Eq, Ord, Show, ...
    It requires a type constructor that takes \textbf{one} parameter

    \begin{alltt}
        > class  Functor    f   where \\
        >       fmap         ::   (a $\to$ b) $\to$ f a $\to$ f b
               \newline
               \newline
        fmap id           =  id \\
        fmap (g . f)      =  fmap g . fmap f
    \end{alltt}
\end{frame}

\begin{frame}
    \frametitle{An Example - Lists}

    \begin{alltt}
        > map :: (a $\to$ b) $\to$ [a] $\to$ [b] \\
        > instance Functor [] where \\
        >   fmap = map
    \end{alltt}

    \begin{block}{Demo!}
        \begin{alltt}
            > fmap (2 +) [1,2,3]
        \end{alltt}
    \end{block}
\end{frame}

\begin{comment}
Further examples

Maybe
-----
> data Maybe x = Just x | Nothing

> instance Functor Maybe where
>   fmap f (Just x) = Just (f x)
>   fmap f Nothing = Nothing


> fmap (++ "ho") (Just "hey")

Trees
-----
> data Tree x = Empty | Node x (Tree x) (Tree x)

> instance Functor Tree where
>   fmap f Empty = Empty
>   fmap f (Node x left right) = Node (f x) (fmap f left) (fmap f right)
\end{comment}

\begin{frame}
    \frametitle{You know this already!!}
    % ruby / javascript examples
    \begin{block}{Ruby}
        \begin{alltt}
            [1, 2, 3].map \{ |n| n * n \}
            % do I put the answers in the slides?
            % 2 slides for this instead of 1 --> with answers?
        \end{alltt}
    \end{block}
    \begin{block}{Javascript}
        \begin{alltt}
            \_.map([1, 2, 3], function(num)\{return num*3;\});
        \end{alltt}
    \end{block}
\end{frame}

\begin{frame}
    \frametitle{The Natural Transformation}
    % Mapping between functors
\begin{alltt}
    alpha            ::  F a -> G a


               alpha\\
         F A ---------> G A\\
           |            |\\
           |            |\\
    fmap f |            | fmap f\\
           |            |\\
           v            v\\
         F B ---------> G B\\
               alpha


Parametricity:\\
alpha . fmap f    =  fmap f . alpha
\end{alltt}
\end{frame}


\section{The end..} % (fold)
\begin{frame}
    \frametitle{The End...}
    \begin{block}{Thanks!}
        Questions?
    \end{block}
\end{frame}

\section{Picture Sources}
\begin{frame}
    \frametitle{Some Sources}
    \begin{itemize}
        \item http://learnyouahaskell.com/
        \item https://en.wikipedia.org/wiki/File:Dining\_philosophers.png
        \item https://commons.wikimedia.org/wiki/File:Spreadsheet.png
    \end{itemize}
\end{frame}

\end{document}
