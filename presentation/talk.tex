\documentclass[12pt, xcolor=table]{beamer}
\usepackage{graphicx}
\usepackage[ngerman]{babel}
\usepackage[utf8]{inputenc}
\usepackage{amsmath}
\usepackage{amssymb}
\usepackage{listings}
\usepackage{hyperref}
\usepackage{fancyvrb}
\usepackage{color}
\usepackage{verbatim}
\usepackage{alltt}

\usepackage[percent]{overpic}
\usepackage[footnotesize, bf]{caption}
\input{theme.tex}
\input{syntax}
\renewcommand{\footnotesize}{\tiny}

\begin{document}
\title{Mapping Between Categories: On Functors in Functional Programming}
\author{Lusy}
\date{\today}

\begin{frame}
    \titlepage
    \begin{block}
        \tiny \url{https://github.com/lusy/functorsTalk}
    \end{block}
\end{frame}

\section{Introduction} % (fold)
\begin{frame}
     \frametitle{Introduction}
     % Find a picture to backup the introductory questions
\end{frame}

\begin{frame}
    \frametitle{Why is functional programming awesome?}
    \begin{itemize}
        \item immutability
        \item concurrency
        \item elegancy
        %\item first class functions
    \end{itemize}
    % code example
    % excel example for the immutability
\end{frame}


\begin{frame}
    \frametitle{Some Maths...}
    What are categories in maths?
    \begin{itemize}
        \item sets
        \item vector spaces
        \item types in programming languages
    \end{itemize}
    %TODO pictures
\end{frame}

\begin{frame}
    \frametitle{Some Maths 2...}
    And we can map between these categories!
    % pictures!
    The mappings are structure preservent...
\end{frame}

\section{Functors Definition}
\begin{frame}
    \frametitle{So what are functors?}
    Structure preservent mappings between categories!
    % example!
\end{frame}

\begin{frame}
    \frametitle{And what are they good for?}
    Motivation

    %\begin{center}
    %    \includegraphics[scale=0.35]{figures/IntroToLDA.png}
    %\end{center}
\end{frame}

\begin{frame}
    \frametitle{The Haskell definition}
    ``Types that can act like a box can be functors. You can think of a list as a box that can be empty or have something inside it, including another box!''
    %pictures!!
\end{frame}

\section{Haskell Functors}
\begin{frame}
    \frametitle{lala}
\end{frame}

\section{The end..} % (fold)
\begin{frame}
    \frametitle{The End...}
    \begin{block}{Thanks!}
        Questions?
    \end{block}
\end{frame}


\end{document}
